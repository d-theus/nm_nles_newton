\documentclass[a4paper, 12pt]{article}
%pack{{{
\usepackage[utf8]{inputenc}
\usepackage[russian]{babel}
\usepackage{amsmath}
%}}}
\newcommand{\argmin}{\mathop{\rm arg~min}\limits}
\begin{document}
\section*{Титульный лист потерялся}
\newpage
\section*{Задание}
\newpage
\tableofcontents
\newpage
\section*{Введение}
\addcontentsline{toc}{section}{Введение}
\newpage
\section*{Определения}
\addcontentsline{toc}{section}{Определения}
\newpage
\section*{Описание методов}
\addcontentsline{toc}{section}{Описание методов}
Будем рассматривать систему нелинейных уравнений:
\[
\left\{
\begin{array}{ccc}	f_1(x) & = &  0\\
	f_2(x) & = &  0\\
	\ldots \\
	f_n(x) & = &  0 
\end{array}
\text{,}
\right.
\]
где $x = (x_1, x_2 \ldots x_n)$\\
Общая проблема методов решения систем нелинейных уравнений заключается в их сугубо локальном
характере сходимости. Это сильно затрудняет их применение в случаях, когда имеются проблемы с 
выбором начального приближения.\\
Для решения данной проблемы используют численные методы оптимизации, а именно, минимизации.
Необходимо поставить задачу минимизации таким образом, чтобы её приближенное решение являлось решением 
исходной системы нелинейных уравнений. Для этого, можно, например, ввести функцию:
\begin{equation*}
	\Phi(x) = (f_1(x))^2 + (f_2(x))^2 + \ldots + (f_n(x))^2 \text{,}
\end{equation*}
находя минимум которой, найдем и решение исходной системы. 

\subsection*{Метод градиентного спуска}
Из математического анализа известно, что функция растет быстрее всего в направлении
своего градиента. Значит, оптимальным направлением движения для минимизации будет направление, 
противоположное градиенту в данной точке. То есть, 
для нахождения последующего приближения нужно выбирать точку, смещенную относительно предыдущего
приближения на вектор антиградиента с неким коэффициентом, большим нуля.
\begin{equation}
	x^{(k+1)} = x^{(k)} - \alpha \nabla \Phi(x^{(k)}) \text{,}
\end{equation}
где $\alpha$, вообще говоря, зависит от текущего приближения, то есть $\alpha = \alpha_k$
\subsection*{Метод наискорейшего спуска}
Итак, известно направление, в котором функция убывает быстрее всего. Однако, нужно
еще определить, как далеко в этом направлении нужно искать следующее приближение.
А оптимальным этот шаг будет, если значение $\Phi(x^{(k+1)})$ минимальное из всех возможных в этом направлении.
То есть 
\begin{equation}
	\alpha_k = \argmin_{\alpha > 0} (\Phi(x^{(k)} - \alpha \nabla \Phi(x^{(k)})))
\end{equation}
Сходимость этого метода линейная, что медленнее, чем, скажем, у метода Ньютона. Однако, как
говорилось выше, метод Ньютона, как и другие, чувствителен к выбору начального приближения.
Используя на начальном этапе метод наискорейшего спуска можно найти хорошее приближение для него.\\
Немного о реализации $argmin$. Использовать будем троичный поиск. Выбранный интервал разбивается на три равных интервала.
Пусть, скажем, изначальный интервал был $[a,b]$. Тогда выбираются точки $m_1 = a+\frac{b-a}{3}$ и $m_2 = b-\frac{b-a}{3}$ и сравниваются значения в них.
Допустим, ищется минимум. Если $f(m_1) > f(m_2)$, то $a = m_1$, в противном случае $b = m_2$. Алгоритм запускается заново с новыми параметрами.
Так продолжается до тех пор, пока $||a - b|| > \epsilon$, где $\epsilon$ --- один из параметров алгоритма --- наименьшая длина отрезка, на котором ищется минимум,при достижении которой поиск останавливается и выбирается точка посередине такого отрезка. В итоге, можно достигнуть точности
\begin{equation*}
	||\argmin_{\alpha}(f(\alpha)) - \frac{a+b}{2}|| \leq \frac{\epsilon}{2}.
\end{equation*}

\subsection*{Метод Ньютона}
%Пусть $A_k$ --- некоторая последовательность невырожденных вещественных $n \times n$-матриц. Тогда, очевидно, последовательность задач
%\begin{equation}
	%x = x - A_k F(x), k=0,1,2,\ldots
%\end{equation}
%имеет те же решения, что и исходная система. Для приближенного нахождения этих решений можно формально 
%записать итерационный процесс 
%\begin{equation}
	%x^{(k+1)} = x^{(k)} - A_k F(x^{(k)}), k=0,1,2,\ldots
%\end{equation}
%\begin{equation}
	%x^{(k+1)} = x^{(k)} + [F'(x^{(k)})]^{-1} F(x^{(k)})
%\end{equation}
Если определено начальное приближение $x^{(0)}$, итерационный процесс нахождения решения системы методом Ньютона можно представить в виде 
\begin{equation}
	x^{(k+1)} = x^{(k)} + \Delta x^{(k)}\text{,}
\end{equation}

где значения $\Delta x^{(k)}$ определяются из решения системы линейных алгебраических уравнений, все коэффициенты которой выражаются
через известное предыдущее приближение $x^{(k)}$. Вектор приращений
\begin{equation*}
	\Delta x^{(k)} = 
	\left(
	\begin{tabular}{c}
		$ \Delta x^{(k)}_1$\\ 
		$ \Delta x^{(k)}_2$\\ 
		$ \Delta x^{(k)}_3$\\ 
		$ \ldots         $ \\ 
		$ \Delta x^{(k)}_n$
	\end{tabular}
	\right)
\end{equation*}
находится из решения уравнения
\begin{equation}
	f(x^{(k)}) + J(x^{(k)}) \Delta x^{(k)} = 0 \text{.}
\end{equation}
Здесь \[J = \left(
	\begin{tabular}{cccc}
		$\frac{\partial f_1(x)}{\partial x_1}$&$\frac{\partial f_1(x)}{\partial x_2}$&$\ldots$&$\frac{\partial f_1(x)}{\partial x_n}$\\
		$\frac{\partial f_1(x)}{\partial x_1}$&$\frac{\partial f_2(x)}{\partial x_2}$&$\ldots$&$\frac{\partial f_2(x)}{\partial x_n}$\\
		$\ldots$&$\ldots$&$\ldots$&$\ldots$\\
		$\frac{\partial f_n(x)}{\partial x_1}$&$\frac{\partial f_n(x)}{\partial x_2}$&$\ldots$&$\frac{\partial f_n(x)}{\partial x_n}$
	\end{tabular}\\
\right) \text{--- Матрица Якоби }\] первых производных вектор-функции $f(x)$.
Выражая из $(4)$ вектор приращений $\Delta x^{(k)}$ и подставляя его в $(3)$, итерационный процесс нахождения решения можно записать в виде 
\begin{equation}
	x^{(k+1)} = x^{(k)} + [J(x^{(k)})]^{-1} f(x^{(k)})\text{.}
\end{equation}
При реализации алгоритма метода Ньютона в большинстве случаев предпочтительным является не вычисление обратной матрицы Якоби, а нахождение 
из системы (4) значений приращений $\Delta x^{(k)}$ и вычисление нового приближения по (3).\\
Использование метода Ньютона предполагает дифференцируемость функций $f_1(x),\ldots ,f_n(x)$ и невырожденность матрицы Якоби. В случае, если 
начальное приближение выбрано в достаточно малой окрестности искомого корня, итерации сходятся к точному решению, причем сходимость квадратичная.\\
В практических вычислениях в качестве условия окончания итераций обычно используется критерий 
\begin{equation}
	|| x^{(k+1)} - x^{(k)} || \leq \epsilon \text{,}
\end{equation}
где $\epsilon$ --- заданная точность.
\newpage
\section*{Листинг программы}
\addcontentsline{toc}{section}{Листинг программы}
\newpage
\section*{Пример работы}
\addcontentsline{toc}{section}{Пример работы}
\newpage
\section*{Заключение}
\addcontentsline{toc}{section}{Заключение}
\newpage
\end{document}
